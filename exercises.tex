% Notes and exercises on Undergraduate Algebra
\documentclass[letterpaper,12pt]{article}
\usepackage{amsmath,amssymb,amsthm,enumitem,fourier}

\newcommand{\Z}{\mathbb{Z}}
\newcommand{\Q}{\mathbb{Q}}

\DeclareMathOperator{\eval}{\mathrm{ev}}

% Theorems
\theoremstyle{definition}
\newtheorem*{exer}{Exercise}

\theoremstyle{remark}
\newtheorem*{rmk}{Remark}

\newtheoremstyle{direction}{0.5em}{0.5em}{}{}{}{}{0.5em}{}
\theoremstyle{direction}
\newtheorem*{fwd}{\(\implies\)}
\newtheorem*{bwd}{\(\impliedby\)}

% Meta
\title{\textit{Undergraduate Algebra}\\Notes and Exercises}
\author{John Peloquin}
\date{}

\begin{document}
\maketitle

% Chapter IV
\section*{Chapter~IV}

% Section 3
\subsection*{\S~3}
\begin{exer}[6]
Let \(K\)~be a subfield of a field~\(E\). Let \(f,g\in K[t]\) with \(f\)~irreducible in~\(K[t]\). Suppose there exists \(\alpha\in E\) such that \(f(\alpha)=0=g(\alpha)\). Then \(f|g\) in~\(K[t]\).
\end{exer}
\begin{proof}
Let \(\eval_{\alpha}:K[t]\to E\) be the homomorphism induced by evaluation at~\(\alpha\) and let \(J=\ker\eval_{\alpha}\). By assumption, we have \(f,g\in J\). Since \(K[t]\)~is a principal ring (Theorem~2.1), \(J=(h)\) for some \(h\in K[t]\). Now in~\(K[t]\), \(h|f\) and \(h|g\), but since \(f\)~is irreducible we must also have~\(f|h\), so~\(f|g\).
\end{proof}

% Section 5
\subsection*{\S~5}
\begin{exer}[4 (Rational root theorem)]
Let \(f(t)=a_nt^n+\cdots+a_0\in\Z[t]\) with \(a_n\ne0\) and \(n\ge1\). If \(f(b/c)=0\) with \(b,c\in\Z\), \(c\ne0\), and \((b,c)=1\), then \(b|a_0\) and \(c|a_n\).
\end{exer}
\begin{proof}
We may assume without loss of generality that \(f\)~is primitive. Since \(f(b/c)=0\), we know \((t-b/c)|f\) in~\(\Q[t]\). But by Gauss (Theorem~5.3), this implies \((ct-b)|f\) in~\(\Z[t]\) since \((b,c)=1\). Therefore \(c|a_n\)~and~\(b|a_0\).
\end{proof}
\begin{rmk}
In particular if \(a_n=1\), then all rational roots of~\(f\) are integral.
\end{rmk}

\begin{exer}[9]
Let \(R\)~be a factorial ring and \(K\)~the quotient field of~\(R\). Let
\[f(t)=t^d+c_{d-1}t^{d-1}+\cdots+c_0\in R[t]\qquad(d\ge2)\]
Let \(p\in R\) be prime and let
\[g(t)=f(t)+p/p^{nd}\in K[t]\qquad(n\ge1)\]
Then \(g\)~is irreducible in~\(K[t]\).
\end{exer}
\begin{proof}
Define \(h(t)=p^{nd}g(t/p^n)\). By direct computation,
\[h(t)=t^d+p^nc_{d-1}t^{d-1}+\cdots+p^{n(d-1)}c_1t+p^{nd}c_0+p\in R[t]\]
By Gauss (Theorem~6.7), it is sufficient to prove that \(h\)~is irreducible in~\(R[t]\). But this follows from Eisenstein (Theorem~6.10) since \(p\)~divides all except the leading coefficient and \(p^2\)~does not divide the constant coefficient.
\end{proof}
\begin{rmk}
Let \(R=\Z\) so \(K=\Q\). Observe that \(g(t)\to f(t)\) as \(n\to\infty\). Therefore there are irreducible polynomials arbitrarily close to~\(f\), with roots arbitrarily close to those of~\(f\). In particular, if \(f\)~has exactly \(d-k\) distinct real roots, then \(g\)~also has exactly \(d-k\) distinct real roots for \(n\)~sufficiently large. (See Exercise~7.)
\end{rmk}

% Chapter VII
\section*{Chapter~VII}

% Section 1
\subsection*{\S~1}
\begin{exer}[11]
Let \(F\)~be a field, \(E\)~a finite extension of~\(F\), and \(F'\)~an arbitrary extension of~\(F\), with \(E\)~and~\(F'\) both contained in some common extension. Then the composite~\(EF'\) is finite over~\(F'\), and
\[[EF':F']\le[E:F]\]
\end{exer}
\begin{proof}
Since \(E/F\)~is finite, it is finitely generated and algebraic (Theorem~1.1). Suppose \(E=F(\alpha)\) with \(\alpha\)~algebraic over~\(F\). We claim \(EF'=F'(\alpha)\). Indeed, by definitions both are the smallest extensions of~\(F\) containing \(F'\)~and~\(\alpha\), so they are equal. Now \(\alpha\)~is trivially algebraic over~\(F'\) since it is algebraic over~\(F\), so \(F'(\alpha)/F'\) is finite and
\[[EF':F']=[F'(\alpha):F']=\deg_{F'}\alpha\le\deg_F\alpha=[F(\alpha):F]=[E:F]\]
since simple algebraic extensions are finite (Theorem~1.3) and the minimal polynomial of~\(\alpha\) over~\(F'\) must divide the minimal polynomial of~\(\alpha\) over~\(F\). The general case \(E=F(\alpha_1,\cdots,\alpha_k)\) now follows by induction and application of the tower law (Theorem~1.4).
\end{proof}

\begin{exer}[12]
Let \(F\)~be a field and let \(E_1\)~and~\(E_2\) be finite extensions of~\(F\) with relatively prime degrees over~\(F\), both contained in some common extension. Then the composite~\(E_1E_2\) is finite over~\(F\) and
\[[E_1E_2:F]=[E_1:F][E_2:F]\]
\end{exer}
\begin{proof}
Since \(E_1E_2/E_2\)~is a translation of the finite extension~\(E_1/F\), it follows that \(E_1E_2/E_2\)~is finite and \([E_1E_2:E_2]\le[E_1:F]\) (Exercise~11). Now by the tower law (Theorem~1.4), \(E_1E_2/F\)~is finite and
\[[E_1E_2:F]=[E_1E_2:E_2][E_2:F]\]
By symmetry,
\[[E_1E_2:F]=[E_1E_2:E_1][E_1:F]\]
Since \([E_1:F]\)~and~\([E_2:F]\) are relatively prime, \([E_1:F]\)~divides~\([E_1E_2:E_2]\), so \([E_1E_2:E_2]=[E_1:F]\) as desired.
\end{proof}
\begin{rmk}
This result shows that translation of a finite extension over a finite extension of relatively prime degree preserves degree. Because degree is an isomorphism type for finite extensions (viewed as finite dimensional vector spaces), this result is analogous to a diamond isomorphism theorem.
\end{rmk}

% Section 3
\subsection*{\S~3}
\begin{exer}[6]
Let \(F\)~be a field of characteristic~\(0\) and \(A\)~an algebraic extension of~\(F\) such that for all \(f\in F[t]\) with \(\deg f\ge1\) there exists \(\alpha\in A\) with \(f(\alpha)=0\). Then \(A\)~is algebraically closed.
\end{exer}
\begin{proof}
Let \(f(t)=a_nt^n+\cdots+a_0\in A[t]\) with \(a_n\ne0\) and \(n\ge1\). Let \(\alpha\)~be a root of~\(f\) in some extension of~\(A\) (Theorem~2.2). We claim \(\alpha\in A\). Observe \(\alpha\)~is algebraic over~\(F\) since \(\alpha\in F(a_1,\ldots,a_n,\alpha)\), an algebraic extension of~\(F\) (Theorem~1.4). Let \(K\)~be a splitting field for the minimal polynomial of~\(\alpha\) over~\(F\) in some extension of~\(A\) (Theorem~3.1) so that \(\alpha\in K\). Since \(F\)~has characteristic~\(0\), \(K=F(\gamma)\) for some primitive element~\(\gamma\in K\) (Theorem~2.5). Now by assumption the minimal polynomial of~\(\gamma\) over~\(F\) has a root~\(\gamma'\in A\), and \(K=F(\gamma')\subseteq A\) since \(K\)~is normal (Theorems~3.3--4). Therefore \(\alpha\in A\) as desired.
\end{proof}

% References
\begin{thebibliography}{0}
\bibitem{lang} Lang, Serge. \textit{Undergraduate Algebra, 3rd ed.} Springer, 2005.
\end{thebibliography}
\end{document}